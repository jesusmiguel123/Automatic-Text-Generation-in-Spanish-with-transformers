%\documentclass[11pt,twocolumn,spanish]{article} 
\documentclass[10pt,twocolumn]{article}
\usepackage[spanish,english]{babel}
\usepackage{indentfirst}
\usepackage{anysize} % Soporte para el comando \marginsize
%\marginsize{1.5cm}{1.5cm}{0.5cm}{1cm}
\marginsize{2,5cm}{1,8cm}{1.5cm}{1,7cm}
\usepackage[psamsfonts]{amssymb}
\usepackage{float}
\usepackage{amssymb}
\usepackage{amsfonts}
\usepackage{amsmath}
\usepackage{amsthm}
\usepackage{multicol}
\usepackage{multirow} 
\usepackage{graphicx}
\usepackage{hyperref}
\usepackage{caption}
\usepackage{subcaption}
\usepackage{tocloft}
\usepackage{natbib}
\usepackage[spanish,es-tabla]{babel}
\usepackage[utf8]{inputenc}
\usepackage{subcaption}
\renewcommand{\cftsecleader}{\cftdotfill{\cftdotsep}}
\renewcommand*\contentsname{Summary}
\renewcommand{\thepage}{}
\theoremstyle{definition}
\renewcommand{\thefootnote}{\fnsymbol{footnote}}

\begin{document}
	
\begin{center}
	%\textbf{Curso:}\\
	\vspace{5pt}
	{\large \textbf{COMO GENERAR TEXTO, UTILIZANDO DIFERENTES MÉTODOS DE DECODIFICACIÓN, PARA LA GENERACIÓN DE LENGUAJES CON TRANSFORMADORES}}\\
	%{\large \textbf{Laboratorio 1} }\\
\end{center}

\begin{center}
	Students:\\
	\vspace{5pt}
	Universidad Nacional de Ingeniería\\
	\vspace{5pt}
	{\large Roberto Alexis Cerna Espiritu }\\
	e-mail: roberto.cerna.e@uni.pe\\
	{\large Abel Alejandro Oliva Valdivia }\\
	e-mail: abel.oliva.v@uni.pe\\
	{\large Franz Rony Ventocilla Tamara }\\
	e-mail: fventocillat@uni.pe\\
	{\large Jesús Miguel Yacolca Huamán }\\
	e-mail: jyacolcah@uni.pe\\
	{\large Eros Aylthon Vargas Torres }\\
	e-mail: evargast@uni.pe\\
	
\end{center}
\vspace{5pt}
%\begin{center}
%	Curso:\\
%	\vspace{5pt}
%	{\large CC0A2 Programación de Dispositivos Móviles}\\
%	{\large Laboratorio 1}\\
%\end{center}
\vspace{20pt}
\begin{abstract*}
{\small
\hspace*{0.5cm}

\begin{center}
    \textbf{Resumen}
\end{center}

\\
La codificación es el proceso mediante el cual la información se convierte en otra forma aceptable para la transmisión. La decodificación invierte este proceso para interpretar la información. En el siguiente trabajo de investigación se realiza un análisis sobre las obtención del texto, basado en la en cinco métodos de decodificación. En primer lugar, se analizaran los datos y se explicará brevemente el funcionamiento de los métodos que usaremos para posteriormente obtener el mensaje codificado.

\textbf{Palabras Clave:} Codificación, Decodificación.
}

\end{abstract*}

\begin{abstract*}
{\small
\hspace*{0.5cm}

\begin{center}
    \textbf{Abstract}
\end{center}

\\
Encryption is the process by which information is converted into another acceptable form for transmission. Decoding reverses this process to interpret the information. In the following research work, an analysis is carried out on obtaining the text, based on the five decoding methods. In the first place, the data will be analyzed and the operation of the methods that we will use to later obtain the encoded message will be briefly explained.

\textbf{Keywords:} Encoding, Decoding.
}

\end{abstract*}

\pagenumbering{arabic}

\tableofcontents

\vspace{20pt}
\hrule
\vspace{10pt}

%\newpage      comienzo     \section{Resumen}


\section{Introducción}

\subsection{Presentación}
\begin{itemize}
    \item dsdsdsds

   
\end{itemize}

\subsection{Objetivos}
% para otras secciones .....\subsubsection{Presentación2}
\begin{itemize}
    \item aqui
    \item aqui
    \item aqui
\end{itemize}

\subsection{Organización del informe}
falta

\subsection{Estado del arte}
falta

\subsection{Aporte de los artículos }
falta

\subsection{Mención de Articulos}
falta

\subsection{Metodología}
falta


\section{Diseño del experimento}

%% para poner en negro     \textbf{¿Pero no íbamos a usar papel?} \\
\subsection{Método : Búsqueda Greedy}
También llamado búsqueda Voraz, es una estrategia de búsqueda por la cual se sigue una heurística consistente en elegir la opción óptima en cada paso local con la esperanza de llegar a una solución general óptima.

\subsection{Método : Beam Search}
También conocido como la estrategia de corte de caminos, mantiene un número predeterminado de las mejores rutas de búsqueda encontradas hasta el momento en un punto dado. Por lo tanto, considera más posibilidades que la primera profundidad de la búsqueda, pero evita el número exponencial de posibilidades de la primera extensión de búsqueda.
Realiza una búsqueda en extensión e incorpora una heurística para escoger en cada nivel solo los mejores nodos. Este método sacrifica completitud a cambio de un enfoque heurístico muy efectivo. Para simplificar, se escogerá múltiples secuencias, siempre tratando de buscar las secuencias con mayor probabilidad y manteniendo el mismo número de secuencias.

\subsection{Método : Sampling}
Este método de muestreo probabilístico sugiere seleccionar datos individuales o de un subconjunto y elegir la palabra que menor probabilidad tenga, esto genera que nuestra elección ya no sea determinista.

\subsection{Método : Top-k Sampling}
El top-k se define como los k elementos más frecuentes dentro de un conjunto de datos y su determinación, este método lo que hace es filtrar las  k palabras más probables y redistribuir las probabilidades entre esas k palabras.

\subsection{Método : Top-P Sampling}
El top-p se define como la probabilidad dentro de un conjunto de datos, en este caso el método va a definir una probabilidad p, entonces se escogerá el subconjunto de menor tamaño de palabras de tal manera que su probabilidad acumulada sea mayor a p.

\section{Experimentos y resultados}
falta

\section{Discusiones}
falta

\section{Conclusiones y trabajos futuros.}
falta

\section{Conclusiones y trabajos futuros.}
falta



\newpage
\section{Bibliografía y referencias}

\begin{itemize}

    \item \url{https://www.bbva.com/es/criptomonedas-sirven-las-monedas-virtuales/}
    \item \url{https://scielo.conicyt.cl/scielo.php?script=sci_arttext&pid=S0719-25842019000100029}
    \item \url{https://criptomoneda.ninja/bitcoin/}
    \item \url{https://www.bbc.com/mundo/noticias-57066481}
    \item \url{https://especiales.dinero.com/bitcoin/index.html}
    \item \url{https://en.wikipedia.org/wiki/Bitcoin}
\end{itemize}
\end{document}

%para imagenes 
%       \begin{figure}[H]
%       \centering
%       \includegraphics[scale=0.95]{3.png}
%       %\caption{Vista activity-main.xml}
%       \end{figure}


